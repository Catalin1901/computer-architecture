\begin{frame}
    \frametitle{Memoria}
    Tipuri de memorie:
    \begin{itemize}
        \item Regiștri
        \item Tampon (Cache)
        \item Principală (RAM)
        \item Memorie auxiliară/externă (HDD, SSD) - Poate să fie parte din subsistemul de intrare/ieșire
    \end{itemize}
    Proprietați:
    \begin{itemize}
        \item Fiecare Locație are același număr de biți
        \item Fiecare locație are o adresă unică 
        \item Spațiul de adresare este omoegen și toate locațiile de memorie sunt echivalente
    \end{itemize}
\end{frame}

\begin{frame}
    \frametitle{Memoria}
    \newsavebox{\asciimemwrite}
    \begin{lrbox}{\asciimemwrite}
        \begin{varwidth}{\maxdimen}
        \VerbatimInput[fontsize=\scriptsize]{media/memwrite.ascii}
        \end{varwidth}
    \end{lrbox}%

    \begin{figure}[h]
        \centering
        \scalebox{0.8}{\usebox{\asciimemwrite}}
    \end{figure}
    \begin{itemize}
        \item Timp acces memorie depinde de lungimea addresei
        \item Timp ciclu de scriere/citire depinde de lungimea cuvântului
    \end{itemize}
\end{frame}


\begin{frame}
    \frametitle{Unitate centrală de prelucrare}
    Componente:
    \begin{itemize}
        \item Unitatea aritmetică și logică (UAL).
        \item Unitatea de comandă.
    \end{itemize}
    Tipuri de UAL:
    \begin{itemize}
        \item UAL cu o singură magistrală.
        \item UAL cu o singură magistrală și un acumulator.
        \item UAL cu trei magistrale.
    \end{itemize}
\end{frame}

\begin{frame}
    \frametitle{Unitatea de comandă}
    \begin{itemize}
        \item Decodifică instrucțiunile și generează semnalele de control.
        \item Preia instrucțiunea de la adresa dată de registru contor program (CP).
        \item Poate să fie implementată:
        \begin{itemize}
            \item Unitate de comandă microprogramată
            \item Unitate de comandă convențională (Automat)
        \end{itemize}
        \item Dependent de setul de instrucțiuni.
    \end{itemize}
\end{frame}

\begin{frame}
    \frametitle{Set de instrucțiuni}
    Tipuri de instrucțiuni:
    \begin{itemize}
        \item Transfer de date
        \item Operații aritmetice
        \item Operații logice și de deplasare
        \item Operații de comparare
        \item Control de flux
    \end{itemize}
    Fiecare instrucțiune are un cod de operație și specificatori de operanzi.
\end{frame}

\begin{frame}
    \frametitle{Subsistem de intrare/ieșire}
    Rolul subsistemului de intrare/ieșire este de a asigura conversia de format și viteză între UCP și dispozitivele periferice.
\end{frame}

\begin{frame}
    \frametitle{Limbaje de programare}
    \newsavebox{\asciiplcomp}
    \begin{lrbox}{\asciiplcomp}
        \begin{varwidth}{\maxdimen}
        \VerbatimInput[fontsize=\scriptsize]{media/plcomp.ascii}
        \end{varwidth}
    \end{lrbox}%

    \begin{figure}[h]
        \centering
        \scalebox{0.8}{\usebox{\asciiplcomp}}
    \end{figure}
    Rolul limbajelor de programare este de a crea iluzia unei ierarhi de calculatoare virtuale.
\end{frame}