\begin{frame}
    \frametitle{Rolurile cadrelor didactice (părere personală)}
\begin{itemize}
    \item De a oferi know-how-ul necesar pentru a vă putea dezvolta în domeniu.
    \item De a fi onest și predictibil.
    \item De a oferi resursele necesare.
    \item De a fi responsabil pentru un mediul sigur.
\end{itemize}

\end{frame}

\begin{frame}
    \frametitle{Implicații}
\begin{itemize}
    \item Nu veți avea întrebări de reproducere a informației.
    \item Nu e un obiectiv să aveți note mici, ci să existe o diferențiere între voi.
    \item Nu voi face witch hunt, dar voi sancționa copierea.
    \item Voi încerca să ofer exerciții model pentru a vă pregăti pentru test/colocviu/examen.
    \item Nu voi accepta comportament abuziv.
\end{itemize}

\end{frame}

\begin{frame}
    \frametitle{Laborator}
    
\begin{itemize}
    \item Schimbarea intervalului asignat în orar este posibilă până la al doilea laborator, dar numai prin schimb cu altă persoană și cu acordul asistenților responsabili de intervalele implicate (nu se pot face schimbări între serii).
    \item Recuperarea unui laborator (pierdut sau în avans) este posibilă doar în limita locurilor disponibile la calculatoarele din sală. Înainte de a vă prezenta pentru recuperarea unui laborator trimiteți un e-mail asistentului vostru și asistentului din intervalul în care vreți să recuperați, pentru a obține acordul lor.
\end{itemize}
\end{frame}


\begin{frame}
    \frametitle{Regulament}
    
    \href{https://cs-pub-ro.github.io/computer-architecture/rules}{https://cs-pub-ro.github.io/computer-architecture/rules}

    Putem aveam "gentlemen's agreement". Propuneri pot veni din partea voastră.
\end{frame}